%%%%%%%%%%%%%%%%%%%%%%%%%%%%%%%%%%%%%%%%%
% Dreuw & Deselaer's Poster
% LaTeX Template
% Version 1.0 (11/04/13)
%
% Created by:
% Philippe Dreuw and Thomas Deselaers
% http://www-i6.informatik.rwth-aachen.de/~dreuw/latexbeamerposter.php
%
% This template has been downloaded from:
% http://www.LaTeXTemplates.com
%
% License:
% CC BY-NC-SA 3.0 (http://creativecommons.org/licenses/by-nc-sa/3.0/)
%
%%%%%%%%%%%%%%%%%%%%%%%%%%%%%%%%%%%%%%%%%

%----------------------------------------------------------------------------------------
%	PACKAGES AND OTHER DOCUMENT CONFIGURATIONS
%----------------------------------------------------------------------------------------

\documentclass[final,hyperref={pdfpagelabels=false}]{beamer}

\usepackage[orientation=portrait,size=a0,scale=1.4,debug]{beamerposter} % Use the beamerposter package for laying out the poster with a portrait orientation and an a0 paper size

\usetheme{I6pd2} % Use the I6pd2 theme supplied with this template

\usepackage[english]{babel} % English language/hyphenation

\usepackage{amsmath,amsthm,amssymb,latexsym} % For including math equations, theorems, symbols, etc

%\usepackage{times}\usefonttheme{professionalfonts}  % Uncomment to use Times as the main font
%\usefonttheme[onlymath]{serif} % Uncomment to use a Serif font within math environments

\boldmath % Use bold for everything within the math environment

\usepackage{booktabs} % Top and bottom rules for tables

\graphicspath{{figures/}} % Location of the graphics files

\usecaptiontemplate{\small\structure{\insertcaptionname~\insertcaptionnumber: }\insertcaption} % A fix for figure numbering

%----------------------------------------------------------------------------------------
%	TITLE SECTION 
%----------------------------------------------------------------------------------------

\title{\huge (2-F-56) Histogram of Gradient Orientations of Signal Plots applied to Brain Computer Interfaces}
\author{Ramele Rodrigo, Santos Juan Miguel, Villar Ana Julia}
\institute[Instituto Tecnológico de Buenos Aires]{Computer Engineering Department,Graduate School of Engineering, Buenos Aires, Argentina}
\date[May. 22nd, 2018]{May. 22nd, 2018}

%----------------------------------------------------------------------------------------
%	FOOTER TEXT
%----------------------------------------------------------------------------------------

\newcommand{\leftfoot}{http://www.LaTeXTemplates.com} % Left footer text

\newcommand{\rightfoot}{john@smith.com} % Right footer text
\newlength{\columnheight}
\setlength{\columnheight}{115cm}
%----------------------------------------------------------------------------------------

\begin{document}

\addtobeamertemplate{block end}{}{\vspace*{2ex}} % White space under blocks

\begin{frame}[t] % The whole poster is enclosed in one beamer frame

\begin{columns}[t] % The whole poster consists of two major columns, each of which can be subdivided further with another \begin{columns} block - the [t] argument aligns each column's content to the top

\begin{column}{.02\textwidth}\end{column} % Empty spacer column

\begin{column}{.465\textwidth} % The first column

%%----------------------------------------------------------------------------------------
%%	OBJECTIVES
%%----------------------------------------------------------------------------------------
%
%\begin{block}{Objectives}
%
%\begin{enumerate}
%\item Donec fringilla, velit id lobortis commodo, eros dui consectetur mi, ut interdum lorem dui sed mauris.
%\item Nulla ac nulla rhoncus est bibendum ullamcorper:
%\item Quisque vestibulum, nisl sit amet gravida ultricies dis parturient montes, nascetur ridiculus musobortis commodo, eros dui consectetur mi.
%\end{enumerate}
%
%\end{block}

%----------------------------------------------------------------------------------------
%	INTRODUCTION
%----------------------------------------------------------------------------------------
            
\begin{block}{Introduction}

\begin{itemize}
\item \textbf{Where are the Waveforms?}
\begin{itemize}
\item Around $71.2\%$ based on Noninvasive EEG Research
\end{itemize}
\item EEG has traditionally focused on temporal waveforms.
\begin{itemize}
\item Bondt-Pompe Permutation Entropy
\item Slope Horizontal Chain Code
\end{itemize}
\item Atlases and guides were developed based on Waveforms.   
\begin{itemize}
\item More interaction is fostered: \textit{"the community asks for a more close relation with clinical inputs"}([5])
\end{itemize}
\end{itemize}

\end{block}

%----------------------------------------------------------------------------------------
%	MATERIALS
%----------------------------------------------------------------------------------------

\begin{block}{Materials}

\begin{columns} % Subdivide the first main column
\begin{column}{.54\textwidth} % The first subdivided column within the first main column
\begin{itemize}
\item Vestibulum nisl, quis euismod velit eros in ligula.
\begin{itemize}
\item Cras rhoncus quam et augue convallis in elementum urna tincidunt.
\end{itemize}
\item Proin ut vestibulum augue.
\begin{itemize}
\item Donec dapibus sagittis neque eu ultrices.
\end{itemize}
\end{itemize}
\end{column}

\begin{column}{.43\textwidth} % The second subdivided column within the first main column
\centering
\begin{figure}
\includegraphics[width=0.8\linewidth]{placeholder.jpg}
\caption{Figure caption}
\end{figure}
\end{column}
\end{columns} % End of the subdivision

\begin{itemize}
\item Curabitur sapien ligula, faucibus in feugiat quis, vestibulum a turpis.
\begin{itemize}
\item Phasellus quis nunc neque. Suspendisse mauris diam, suscipit non gravida in, placerat id enim. Ut nec ipsum in lectus ultrices sagittis.
\item Ut nec ipsum in lectus ultrices sagittis.
\item Phasellus quis nunc neque.
\end{itemize}
\end{itemize}

\end{block}

%----------------------------------------------------------------------------------------
%	METHODS
%----------------------------------------------------------------------------------------

\begin{block}{Methods}
\begin{itemize}
\item Signal Plottting
\begin{itemize}
\item Cum sociis natoque penatibus et magnis dis parturient montes, nascetur ridiculus mus. 
\item Proin in nisi diam.
\item Nam ultricies pellentesque nunc, ultrices volutpat nisl ultrices a.
\end{itemize}

\item Volutpat 
\begin{itemize}
\item Duis semper lorem eget dui dignissim porttitor.
\item Nulla facilisi. In ullamcorper lorem quis dolor.
\end{itemize}


\item The Histogram of Gradient Orientations
\begin{itemize}
\item Suspendisse potenti. Fusce a est eget turpis rhoncus varius sed sed dui. Cras justo nibh, bibendum a cursus eget, consequat et dui. Maecenas vel nisl elit, sed dignissim dolor. 
\item In hac habitasse platea dictumst.
\end{itemize}

\begin{align*}
 h(\theta,i,j) = 3 s \sum_{\mathbf{p}} w_\mathrm{ang}(\angle J(\mathbf{p}) - \theta)\, w_{ij}\left(\frac{\mathbf{p} - \mathbf{kp}}{3 s}\right)\, |J(\mathbf{p})|
\label{eq:histogram}
\end{align*}

\end{itemize}

\end{block}

%----------------------------------------------------------------------------------------
%	MATHEMATICAL SECTION
%----------------------------------------------------------------------------------------

\begin{block}{Classification}

\begin{itemize}
\item Maecenas Ultricies Feugiat Velit Non Mattis.
\begin{itemize}
\item Duis ante erat, bibendum nec tempus nec, interdum quis est. Nulla at mollis tortor. Phasellus quis leo dolor, aliquam laoreet orci $X$ Donec dapibus sagittis neque eu nec, interdum quis est. $Y_n, n=1,\cdots,N$ ndum nec tempus nec, interd
\begin{align*}
X \rightarrow r(X) & = \arg \max_{c} \Big\{ \max_n \big\{ \sum_{x_i \in X} \delta(x_i,Y_{n,c})\big\} \Big\} 
\end{align*}
\item Cras faucibus scelerisque cursus. Proin ut vestibulum augue. $\delta(x_i,Y_{n,c})$
\end{itemize}
\item Fusce tempus arcu id ligula varius dictum. Donec ut nisl dui, ac consectetur elit. In nec enim porta augue venenatis sollicitudin. Phasellus quis nunc neque. Suspendisse mauris diam, suscipit non gravida in, placerat id enim. Ut nec ipsum in lectus ultrices sagittis.
\end{itemize}

\end{block}

%----------------------------------------------------------------------------------------

\end{column} % End of the first column

\begin{column}{.03\textwidth}\end{column} % Empty spacer column
 
\begin{column}{.465\textwidth} % The second column

%----------------------------------------------------------------------------------------
%	RESULTS
%----------------------------------------------------------------------------------------

\begin{block}{Results: Table}

\begin{itemize}
\item Transient and oscillatory phenomena
\end{itemize}

\begin{table}
\begin{tabular}{l l l}
\toprule
\textbf{Treatments} & \textbf{Response 1} & \textbf{Response 2}\\
\midrule
Treatment 1 & 0.0003262 & 0.562 \\
Treatment 2 & 0.0015681 & 0.910 \\
Treatment 3 & 0.0009271 & 0.296 \\
\bottomrule
\end{tabular}
\caption{Table caption}
\end{table}

\begin{itemize}
\item Sollicitudin Vel Orci
\item Maecenas Ultricies Feugiat Velit Non Mattis.
\end{itemize}

\begin{table}
\begin{tabular}{l l l}
\toprule
\textbf{Treatments} & \textbf{Response 1} & \textbf{Response 2}\\
\midrule
Treatment 1 & 0.0003262 & 0.562 \\
Treatment 2 & 0.0015681 & 0.910 \\
Treatment 3 & 0.0009271 & 0.296 \\
\bottomrule
\end{tabular}
\caption{Table caption}
\end{table}
     
\end{block}

%------------------------------------------------

\begin{block}{Results: Figure}

\begin{figure}
\includegraphics[width=0.8\linewidth]{placeholder.jpg}
\caption{Figure caption}
\end{figure}

\end{block}

%----------------------------------------------------------------------------------------
%	CONCLUSION
%----------------------------------------------------------------------------------------

\begin{block}{Significance}

\begin{itemize}
\item A method which is biomimetically based on how the visual cortex works by detecting orientations, ironically, is used preciely to detect information from the brain.
\item It has universal applicability because the same basic methodology can be applied to detct different patterns in EEG for BCI
\item It has the potential to foster close collaboration with physicians and electroencephalograph technicias
\item Follows the established procedure of the clinical EEG of analyzing waveforms by their shapes.
\item Eases the clinical acceptance and use of qEEG technologies.
\end{itemize}

\end{block}

%----------------------------------------------------------------------------------------
%	REFERENCES
%----------------------------------------------------------------------------------------

\begin{block}{References}
              \begin{itemize}
                \item [1] \small Guger C., Allison B.Z., Lebedev M.A. (2017) Recent Advances in Brain-Computer Interface Research—A Summary of the BCI Award 2016 and BCI Research Trends.
                \item [2] \small Schomer, D.L.; Silva, F.L.D. Niedermeyer’s Electroencephalography: Basic Principles, Clinical Applications, and Related Fields. 2010.
                \item [3] Berger, S.; Schneider, G.; Kochs, E.; Jordan, D. Permutation Entropy: Too Complex a Measure for EEG Time Series? 2017.
                \item [4] Alvarado-González, M.; Garduño, E.; Bribiesca, E.; Yáñez-Suárez, O.; MedinaBañuelos, V. P300 Detection Based on EEG Shape Features. 2016
                \item [5] Yamaguchi, T.; Fujio, M.; Inoue, K.; Pfurtscheller, G. Design Method of Morphological Structural Function for Pattern Recognition of EEG Signals During Motor Imagery and
Cognition. 2009
                \item [6] Ramele, R.; Villar, A.J.; Santos, J.M. BCI classification based on signal plots and SIFT descriptors. 2016;
                \item [7] D. G. Lowe, Object recognition from local scale-invariant features.1999
              \end{itemize}
                      
%\nocite{*} % Insert publications even if they are not cited in the poster
%\small{\bibliographystyle{unsrt}
%\bibliography{sample}}

\end{block}

%----------------------------------------------------------------------------------------
%	ACKNOWLEDGEMENTS
%----------------------------------------------------------------------------------------

\begin{block}{Acknowledgments}

\begin{itemize}
\item This project was supported by the ITBACyT-15 funding program issued by ITBA University in Buenos Aires, Argentina
\end{itemize}

\end{block}

%----------------------------------------------------------------------------------------
%	CONTACT INFORMATION
%----------------------------------------------------------------------------------------


\begin{block}{Contact Information}

\begin{itemize}
\item Web: \href{http://www.university.edu/smithlab}{http://www.university.edu/smithlab}
\item Email: \href{mailto:john@smith.com}{john@smith.com}
\item Phone: +1 (000) 111 1111
\end{itemize}

\end{block}

%----------------------------------------------------------------------------------------

\end{column} % End of the second column

\begin{column}{.015\textwidth}\end{column} % Empty spacer column

\end{columns} % End of all the columns in the poster

\end{frame} % End of the enclosing frame

\end{document}